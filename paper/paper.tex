\documentclass{scrartcl}
\usepackage[utf8]{inputenc}
\usepackage[T1]{fontenc}

% Für schöneren Blocksatz
\usepackage{microtype}
% Für deutsche Sprache
\usepackage[ngerman]{babel}
% Für deutsche Zitate / Anführungszeichen
\usepackage{csquotes}
% Für klickbare links (Kasten werden nicht mitgedruckt)
\usepackage{hyperref}
% Für Quellenangaben
\usepackage{biblatex}
\addbibresource{sources.bib}

\title{Erkennung ereigniskorellierter Potenziale eines Elektroenzephalogramms durch eine KI }
\date{\today}
\author{Alexander Reimer \and Matteo Friedrich}
\begin{document}
	\maketitle
    \begin{abstract}
		In diesem Projekt wollen wir einen Roboter mithilfe bloßer Gedankenkraft steuern.
		
		Um Daten über das Gehirn zu bekommen, nutzen wir einen Elektroenzephalographen, kurz EEG, welches durch Elektroden an der Kopfhaut die Spannungen innerhalb des Gehirns misst. Dies werten wir mithilfe eines Neuronalen Netzes aus, welches wir vorher darauf trainiert haben, Muster in diesen Daten zu erkennen. So können wir bestimmte Ereignisse anhand der EEG-Daten ableiten, z.B. ob jemand geblinzelt hat oder sich gerade konzentriert.
		
		Das Ziel ist es dann, durch das Erkennen verschiedener dieser sogenannten ereigniskorrelierten Potentialen (EKPs) einen Roboter nur mit Gedanken steuern zu können. 
	\end{abstract}

	\newpage

	\tableofcontents

	\newpage

	\section{Einleitung}
	In diesem Projekt wollen wir einen Roboter mithilfe bloßer Gedankenkraft steuern.
	\section{Methode und Vorgehensweise}

	\subsection{Materialien} \label{Materialien}

	\begin{itemize}
		\item EEG
		\begin{itemize}
			\item 4 Channel Ganglion Board von OpenBCI $\star$
			\item 2x Spike Electrodes  $\star$
			\item 2x Flat Electrodes $\star$
			\item Electrode Gel 
		\end{itemize}

		\item Roboter
		\begin{itemize}
			\item Lego Mindstorms EV3 Brick
			\item 2x EV3 großer Motor
			\item Raspberry Pi 4B 8GB
			\item Diverse Legoteile
		\end{itemize}

		\item Software
		\begin{itemize}
			\item Flux.jl für das neuronale Netz
			\item BrainFlow.jl als Schnittstelle zum EEG
			\item FFTW.jl für die Fast Fourier Transformation
			\item CUDA.jl zum effektiven Nutzen einer NVIDIA GPU
			\item PyPlot.jl zum plotten
			\item BSON.jl zum Speichern und Laden von Netzwerken
		\end{itemize}
	\end{itemize}

	\subsection{Vorgehensweise}

	Die Idee eines BCI -- Brain-Computer Interface -- ist nicht neu, sondern wird intensiv erforscht. Aufgrund bereits durchgeführter Experimente wissen wir, dass es möglich ist. Besonders informativ bei unserer Recherche war der Artikel \enquote{Brain-computer interfaces for communication and rehabilitation} \cite{BCIChaudhary}.

	Wir hoffen, neben dem Erlangen von Erfahrung in diesem interessanten Bereich auch selbst dazu beizutragen. Dies wollen wir erreichen durch 
	\textit{a)} eine allgemeine Anwendung, bei der kein vorheriges Trainieren für eine fremde Person benötigt wird, \textit{b)} das Verwenden von günstiger, für viele bezahlbare Hardware, sowie \textit{c)} ein performantes Programm welches \textit{d)} leicht für die eigenen Zwecke anpassbar ist.

	Unser Projekt lässt sich grob in 3 Teile unterscheiden. 

	Zum einem gibt es den neurobiologischen Teil. Dieser besteht aus der Messung von Gehirnaktivität und der Umwandlung dieser Aktivität in für uns nutzbare Daten. 

	Der zweite Teil besteht aus der Verarbeitung dieser Signale. Hierfür nutzen wir ein neuronales Netz, welches Muster in den Gehirnaktivitäten erkennen kann. 

	Der letzte Teil von unserem Projekt beinhaltet die Konstruktion und Steuerung eines EV3 Roboters. Je nachdem, was das Neuronale Netz ausgibt, soll sich dieser Roboter anders verhalten und so über Gedanken steuerbar sein. 


	\subsubsection{Elektroenzephalographie}

	Bei der Elektroenzephalographie werden Elektroden an der Kopfoberfläche platziert. Diese können die sehr kleinen Spannungen messen, die durch Reize im Gehirn entstehen und durch den Schädel dringen. Diese Spannungen werden nicht von einzelnen Nervenzellen erzeugt, sondern geben die Summe aller lokalen Spannungen wieder. Man kann also auch nur ungefähr, je nach Anzahl der vorhandenen Elektroden, sagen, wo genau im Gehirn ein bestimmter Reiz ausgelöst wurde. 

	Wir untersuchen dabei Ereigniskorrelierte Potentiale (EKPs). Dies sind bestimmte Spannungsschwankungen (\enquote{Potentiale}), welche in Zusammenhang mit einem (beobachtbaren) Ereignis stehen, wie z.B. Blinzeln oder Armbe. 
	
	Weiter ist es möglich, nur durch Gedanken eine Steuerung auszuführen. Dies funktioniert jedoch in den von uns gefundenen Beispielen meist durch Instrumentelle oder Klassische Konditionierung, also durch das Bestrafen und Belohnen auf Basis der Messungen des EEG. So kann das Gehirn darauf trainiert werden, auf Verlangen bestimmte, vorher festgelegte Aktivität auszulösen, die dann gemessen und ausgewertet werden kann. 
	
	Wir wollen dies nicht machen, da die Konditionierung Zeit benötigt, nicht unser Ziel einer allgemeinen Anwendbarkeit erfüllt, und voraussichtlich bessere Ausrüstung erfordert als wir haben. 
	
	Jedoch wollen wir in Zukunft versuchen, mithilfe unseres Neuronalen Netzes bereits bei allen Menschen vorhandene, selbstkontrollierbare Gehirnaktivität zu finden und sicher zu erkennen. Solche Gehirnaktivität könnte z.B. der allgemeine Gedanke an \enquote{Rechts} sein, was womöglich mit erhöhter Aktivität auf der linken Hirnhälfte in Verbindung stehen könnte. 
	
	Zuerst probieren wir es aber mit EKPs, da diese deutlich leichter zu erkennen sind und wir so erstmal das Konzept eines Neuronalen Netzes in einem solchen Kontext ausprobieren können. Denn wenn wir EKPs noch nicht sicher erkennen könnten, wären komplexere Zusammenhänge erst recht nicht möglich.
	 
	\textit{(Hier eine Abbildung von einem EEG mit Blinzeln) }
	
	Unser EEG-Gerät, das Ganglion Board, hat vier Elektroden, mit einer zeitlichen Auflösung von jeweils 200 Herz (200 Messungen pro Sekunde). Diese Elektroden haben wir am Okzipitallappen platziert. Grund dafür ist, dass wir mit unserem Gerät primär Alphawellen (Wellen im Bereich zwischen 7 und 13 Herz) gut erkennen können und dieser dort besonders ausgeprägt sind.
	
	Unser EEG-Gerät besteht aus einem Klettband mit Löchern für die Elektroden, einer Platine, in welche die Elektroden eingesteckt werden, und einem USB-Dongle, mit welchem die Signale der Platine kabellos empfangen werden können.
	
	Um die Signale in Julia empfangen, in Dateiein speichern, und laden zu können, haben wir BrainFlow benutzt.

	\subsubsection{Neuronales Netz}

	Um ein neuronales Netzwerk zu erstellen haben wir Flux benutzt. Wir haben immer wieder mit der Struktur experimentiert, momentan ist sie … 

	Zum Trainieren haben wir Supervised Learning benutzt.

	\subsubsection{Roboter}

	\section{Ergebnisse}

	Zuerst haben wir versucht, eine KI zu trainieren, welche erkennen kann, ob eine Person gerade geblinzelt hat oder nicht. Diese könnte man zum Beispiel nutzen, indem man einen Roboter immer dann nach vorne fahren lässt, wenn eine Versuchsperson blinzelt. Hierbei gibt es zwei sofort erkennbare Probleme. Zum einen ist Blinzeln ein Menschlicher Reflex, den man nicht abstellen kann. Das heißt, dass es in unserem Beispiel mit dem Roboter nicht möglich wäre den Roboter langfristig stehen bleiben zu lassen, weil man irgendwann blinzeln müsste. Außerdem erkennen wir beim Blinzeln keine Gehirnaktivitäten, sondern nur … 

	\section{Diskussion}

	\section{Danksagung}

	\subsection{Finanzierung} \label{Foerderverein}

	Danke an den Förderverein \enquote{Gesellschaft der Freunde des Gymnasium Eversten e.V.} für die Finanzierung des EEG-Geräts. Alle finanzierten Teile sind in der \hyperref[Materialien]{Materialliste} mit $\star$ gekennzeichnet. 

	\section{Quellen \& Referenzen}

	\printbibliography{}
\end{document}